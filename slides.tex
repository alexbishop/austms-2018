\documentclass[11pt]{beamer}

\usepackage[utf8]{inputenc}
\usepackage[T1]{fontenc}
\usepackage{lmodern}
\usepackage[australian]{babel}

\usepackage[style=verbose,citestyle=authortitle-comp]{biblatex}

\addbibresource{slides.bib}
\usepackage{xpatch}
\xapptobibmacro{cite}{\setunit{\nametitledelim}\printfield{year}}{}{}
%\AtBeginBibliography{\footnotesize}
\let\oldfootnotesize\footnotesize
\renewcommand*{\footnotesize}{\oldfootnotesize\tiny}

\usepackage{csquotes}

\usepackage{amsthm,amsfonts,amssymb,amsmath,mathtools,graphicx}

\usepackage{microtype}

\usetheme{Copenhagen}


\usefonttheme[onlymath]{serif}
\setbeamertemplate{navigation symbols}{}

\begin{document}

\author{Alex Bishop\\\vspace{.6em}Supervisor: A/Prof Murray Elder}
%\title{Bounded Automata Groups are co-ET0L}
\title{Formal Languages in Group co-Word Problems}
%\subtitle{}
%\logo{}
\institute{University of Technology Sydney}
\date{Wednesday 5 December 2018}%16:30 Wed 5 Dec
%\subject{}
%\setbeamercovered{transparent}
%\setbeamertemplate{navigation symbols}{}
\begin{frame}[plain]
	\maketitle
\end{frame}

%%%%%%%%%%%%%%%%%%%%%%%%%%%%%%%%%%%%%%%%%%%%%%%%%%%%%%%%%%%%%%%%%%%
%%%%%%%%%%%%%%%%%%%%%%%%%%%%%%%%%%%%%%%%%%%%%%%%%%%%%%%%%%%%%%%%%%%
%%%%%%%%%%%%%%%%%%%%%%%%%%%%%%%%%%%%%%%%%%%%%%%%%%%%%%%%%%%%%%%%%%%

\begin{frame}
\frametitle{Funding Acknowledgement}
\begin{center}
	\includegraphics[width=0.8\linewidth]{figure/austms}
	
	\vspace{2em}
	
	\large
	This talk is funded by the AustMS Student Support Scheme
\end{center}
\end{frame}

%%%%%%%%%%%%%%%%%%%%%%%%%%%%%%%%%%%%%%%%%%%%%%%%%%%%%%%%%%%%%%%%%%%
%%%%%%%%%%%%%%%%%%%%%%%%%%%%%%%%%%%%%%%%%%%%%%%%%%%%%%%%%%%%%%%%%%%
%%%%%%%%%%%%%%%%%%%%%%%%%%%%%%%%%%%%%%%%%%%%%%%%%%%%%%%%%%%%%%%%%%%

\begin{frame}
\begin{center}
	{\LARGE
	
	\url{https://arxiv.org/abs/1811.10157}
	
}
\pause
\vspace{1.5cm}
	
	\textbf{\Large Bounded Automata Groups are co-ET0L}
	
\end{center}
	{
		\textit{Abstract:}
		Holt and R\"over proved that finitely generated bounded automata groups have indexed co-word problem. Here we sharpen this result to show they are in fact co-ET0L.
	}
	
	\phantom{\cite{holt2006}}
	
%	\only<2->{
%		
%		\vspace{1em}
%		
%		\emph{or}
%		
%		\vspace{1em}
%		
%		\url{https://alexbishop.github.io}
%	}
\end{frame}

%%%%%%%%%%%%%%%%%%%%%%%%%%%%%%%%%%%%%%%%%%%%%%%%%%%%%%%%%%%%%%%%%%%
%%%%%%%%%%%%%%%%%%%%%%%%%%%%%%%%%%%%%%%%%%%%%%%%%%%%%%%%%%%%%%%%%%%
%%%%%%%%%%%%%%%%%%%%%%%%%%%%%%%%%%%%%%%%%%%%%%%%%%%%%%%%%%%%%%%%%%%

%\begin{frame}
%\frametitle{Group}
%
%\begin{block}{Definition: Group}
%
%A group $G$ is such that for each $a,b,c \in G$ we have:
%
%\begin{itemize}
%	\item \makebox[6.2em][l]{\textbf{Closure:}}
%	    $a \cdot b \in G$ 
%	\item \makebox[6.2em][l]{\textbf{Associativity:}}
%	    $a \cdot b \cdot c = (a \cdot b) \cdot c = a \cdot (b \cdot c)$
%	\item \makebox[6.2em][l]{\textbf{Identity:}}
%		$1 \cdot a = a \cdot 1 = a$
%	\item \makebox[6.2em][l]{\textbf{Inverses:}}
%		$a \cdot a^{-1} = a^{-1} \cdot a = 1$
%\end{itemize}
%
%\end{block}
%
%\begin{block}<2->{Definition: Finitely Generated Group}
%
%A group is finitely generated, if there exists a finite $X \subseteq G$, such that for each $a \in G$ we have some $y_1, y_2,\dots, y_n \in X \cup X^{-1}$ so that
%\[
%	a = y_1 \cdot y_2 \cdots y_n.
%\]
%Then, $X$ is called a generating set.
%
%\end{block}
%
%\end{frame}

%%%%%%%%%%%%%%%%%%%%%%%%%%%%%%%%%%%%%%%%%%%%%%%%%%%%%%%%%%%%%%%%%%%
%%%%%%%%%%%%%%%%%%%%%%%%%%%%%%%%%%%%%%%%%%%%%%%%%%%%%%%%%%%%%%%%%%%
%%%%%%%%%%%%%%%%%%%%%%%%%%%%%%%%%%%%%%%%%%%%%%%%%%%%%%%%%%%%%%%%%%%

\begin{frame}
\frametitle{Co-Word Problems}

\begin{block}{Definition: Co-Word Problem}
Given a group $G$ with finite generating set $X$, the co-word problem
\[
	\mathrm{coW}(G,X)
	\coloneqq
	\left\{
	w \in \left(X \cup X^{-1} \right)^\star
	\, :\,
	w \neq_G 1
	\right\}
\]
is the set of all words not representing $1 \in G$.
\end{block}

\end{frame}


\begin{frame}[t]
\frametitle{Co-Word Problems}
\framesubtitle{\large Previously known}

\begin{definition}
\[
\mathrm{coW}(G,X)
\coloneqq
\left\{
w \in \left(X \cup X^{-1} \right)^\star
\, :\,
w \neq_G 1
\right\}
\]
\end{definition}

\vspace*{1em}

It is natural to consider the computational complexity of such sets by their formal language class.
For example, a group has

\begin{itemize}
	\item
	regular co-word problem iff it's finite\footcite{anisimov1971}; and
	
	\item
	deterministic context-free co-word problem iff it's virt.\@ free\footcite{muller1983}\textsuperscript{,}\footcite{muller1985}
	
	\item
	deterministic $n$-counter iff it is virt.\@ $\mathbb{Z}^n$\!\!\!\phantom{m}\footcite{elder2008}
\end{itemize}

\end{frame}



\begin{frame}[t]
\frametitle{Co-Word Problems}
\framesubtitle{\large Previously known \textit{cont.}}


\begin{definition}
	\[
	\mathrm{coW}(G,X)
	\coloneqq
	\left\{
	w \in \left(X \cup X^{-1} \right)^\star
	\, :\,
	w \neq_G 1
	\right\}
	\]
\end{definition}

\vspace*{1em}

It is also known that,

\begin{itemize}
	\item finitely generated subgroups of Thompson's group $V$ are have context-free co-word problem\footcite{lehnert2007}
	
	\item bounded automata groups have indexed co-word problem\footcite{holt2006}
\end{itemize}

\end{frame}


\begin{frame}[t]
\frametitle{Co-Word Problems}
\framesubtitle{\large Invariance under Generating set}

\begin{definition}
	\[
	\mathrm{coW}(G,X)
	\coloneqq
	\left\{
	w \in \left(X \cup X^{-1} \right)^\star
	\, :\,
	w \neq_G 1
	\right\}
	\]
\end{definition}

\vspace*{1em}

We are only interested in a class $\mathcal{C}$ of formal language language if it is closed under inverse word homomorphism.
\pause
\[
	\varphi: X \to Y^\star
\]
which we extend to a map from $X^\star \to Y^\star$.

\end{frame}

\begin{frame}[t]
\frametitle{Co-Word Problems}
\framesubtitle{\large Invariance under Generating set}

\begin{definition}
	\[
	\mathrm{coW}(G,X)
	\coloneqq
	\left\{
	w \in \left(X \cup X^{-1} \right)^\star
	\, :\,
	w \neq_G 1
	\right\}
	\]
\end{definition}

\vspace*{1em}

We are only interested in a class $\mathcal{C}$ of formal language language if it is closed under inverse word homomorphism.
\[
\varphi: X \to Y^\star
\]
which we extend to a map from $X^\star \to Y^\star$.

\ \\

For such classes, if $\mathrm{coW}(G,X)$ is in $\mathcal{C}$, then
$\mathrm{coW}(G,Y)$ is in $\mathcal{C}$ for each finite generating set $Y$.\footcite{holt2005}

\end{frame}

%\begin{frame}
%\frametitle{Languages}
%\begin{center}
%	\includegraphics[width=\linewidth]{figure/lang}
%\end{center}
%\end{frame}

%\begin{frame}
%\frametitle{Regular Co-Word Problem}
%give an example of a finite-state machine which corresponds to the co-word problem of a finite group.
%\end{frame}

%%%%%%%%%%%%%%%%%%%%%%%%%%%%%%%%%%%%%%%%%%%%%%%%%%%%%%%%%%%%%%%%%%%
%%%%%%%%%%%%%%%%%%%%%%%%%%%%%%%%%%%%%%%%%%%%%%%%%%%%%%%%%%%%%%%%%%%
%%%%%%%%%%%%%%%%%%%%%%%%%%%%%%%%%%%%%%%%%%%%%%%%%%%%%%%%%%%%%%%%%%%

\begin{frame}
\frametitle{Bounded Automata Groups}
\framesubtitle{\large Automorphisms of a Rooted Tree $\mathrm{Aut}(\mathcal{T}_d)$}
\hspace{2em}\includegraphics[width=.8\linewidth]{figure/labelledTree}
\end{frame}

\begin{frame}
\frametitle{Bounded Automata Groups}
\framesubtitle{\large Automorphisms of a Rooted Tree $\mathrm{Aut}(\mathcal{T}_d)$}
\hspace{2em}\includegraphics[width=.8\linewidth]{figure/labelledTree1}
\end{frame}

\begin{frame}
\frametitle{Bounded Automata Groups}
\framesubtitle{\large Automorphisms of a Rooted Tree $\mathrm{Aut}(\mathcal{T}_d)$}
\hspace{2em}\includegraphics[width=.8\linewidth]{figure/labelledTree2}
\end{frame}

\begin{frame}
\frametitle{Bounded Automata Groups}
\framesubtitle{\large Automata Automorphism}
\hspace{2em}\includegraphics[width=.8\linewidth]{figure/girgb}
\end{frame}

\begin{frame}
\frametitle{Bounded Automata Groups}
\framesubtitle{\large Automata Automorphism}
\hspace{2em}\includegraphics[width=.8\linewidth]{figure/girgb2}
\end{frame}


\begin{frame}
\frametitle{Bounded Automata Groups}
\framesubtitle{\large Automata Automorphism}
\hspace{2em}\includegraphics[width=.8\linewidth]{figure/girgb3}
\end{frame}

\begin{frame}
\frametitle{Bounded Automata Groups}
\framesubtitle{\large Automata Automorphism}
\hspace{2em}\includegraphics[width=.8\linewidth]{figure/girgb4}
\end{frame}

\begin{frame}
\frametitle{Bounded Automata Groups}
\framesubtitle{\large Automata Automorphism}
\hspace{2em}\includegraphics[width=.8\linewidth]{figure/girgb5}
\end{frame}

%\begin{frame}
%\frametitle{Bounded Automata Groups}
%\framesubtitle{\large Automata Automorphism}
%\hspace{2em}\includegraphics[width=\linewidth]{figure/girgorchuk}
%\end{frame}

\begin{frame}
\frametitle{Bounded Automata Groups}
\framesubtitle{\large Bounded Automaton Automorphisms $\mathcal{D}(\mathcal{T}_d)$}
\begin{definition}
A tree automorphism $\alpha \in \mathrm{Aut}(\mathcal{T}_d)$ is called \emph{bounded} if there exists a constant $N = N(\alpha) \in \mathbb{N}$ such that, for any level of the tree $\mathcal{T}_d$, the action of $\alpha$ is trivial for all but up-to $N$ sub-trees.
\end{definition}

\begin{itemize}
	\item<2-> We call a group bounded automata if it all its elements are bounded automaton automorphisms
	
	\item<3-> It is enough to prove that the generating set only contains bounded automaton automorphisms
\end{itemize}

\end{frame}

\begin{frame}
\frametitle{Bounded Automata Groups}
\framesubtitle{\large Finitary Automorphisms $\mathrm{Fin}(\mathcal{T}_d)$}

\includegraphics[width=.45\linewidth]{figure/finitaryA}
\hfill
\includegraphics[width=.45\linewidth]{figure/finitaryB}

\end{frame}


\begin{frame}
\frametitle{Bounded Automata Groups}
\framesubtitle{\large Directed Automorphisms $\mathrm{Dir}(\mathcal{T}_d)$}

\centering

\includegraphics[height=4.5cm]{figure/directedX}
\hspace*{-3em}
\includegraphics[height=4.5cm]{figure/directedY}
\includegraphics[height=4.5cm]{figure/directedZ}

\end{frame}


\begin{frame}
\frametitle{Bounded Automata Groups}

\begin{block}{Theorem: (Sidki; 2000)}
	The group of all bounded automata automorphisms is generated by the set of finitary and directed automata automorphisms.
\end{block}
\phantom{\footcite{sidki2000}}

Let $G \subseteq \mathcal{D}(\mathcal{T}_d)$ with finite generating set $X$.
We can define a map
\vspace{-1em}
\[
\varphi:
X
\to
\left(
\mathrm{Fin}(\mathcal{T}_d) \cup \mathrm{Dir}(\mathcal{T}_d)
\right)^\star
\]

\vspace{-1em}
so that $x =_{\mathcal{D}(\mathcal{T}_d)} \varphi(x)$ for each $x \in X$.
Let
\vspace{-1em}
\begin{multline*}
Y
=
\Big\{
\alpha \in \mathrm{Fin}(\mathcal{T}_d) \cup \mathrm{Dir}(\mathcal{T}_d) 
\,:\,
\alpha \text{ is a factor of } \varphi(x)\\ \text{ for some } x \in X
\Big\}.
\end{multline*}

\vspace{-1em}
Then, $\mathrm{coW}(G,X)$ is the inverse image of $\mathrm{coW}(G,Y)$ w.r.t.\@ $\varphi$.

\end{frame}

%%%%%%%%%%%%%%%%%%%%%%%%%%%%%%%%%%%%%%%%%%%%%%%%%%%%%%%%%%%%%%%%%%%
%%%%%%%%%%%%%%%%%%%%%%%%%%%%%%%%%%%%%%%%%%%%%%%%%%%%%%%%%%%%%%%%%%%
%%%%%%%%%%%%%%%%%%%%%%%%%%%%%%%%%%%%%%%%%%%%%%%%%%%%%%%%%%%%%%%%%%%

\begin{frame}
\frametitle{ET0L}

We have two \emph{alphabets},

\begin{itemize}
	\item \makebox[0.9em][l]{$\Sigma$} --- alphabet of terminals; and
	\item \makebox[0.9em][l]{$V$}      --- alphabet of non-terminals.
\end{itemize}

\visible<2->{We define a \emph{table} as finite set of variable replacements}
\begin{example}<3->
Let $\Sigma = \{a,b\}$ and $V = \{S,A,B\}$, then the following are tables
\[
\alpha \, : \,
	\left\{
	\begin{aligned}
		S &\to SS\\
		S &\to S\\
		S &\to AB\\
		A &\to A\\
		B &\to B
	\end{aligned}
	\right.
	\qquad
\beta \, : \,
	\left\{
	\begin{aligned}
		S &\to S\\
		A &\to aA\\
		B &\to bB
	\end{aligned}
	\right.
	\qquad
\gamma \, : \,
	\left\{
	\begin{aligned}
		S &\to S \\
		A &\to \varepsilon \\
		B &\to \varepsilon
	\end{aligned}
	\right.
\]
\end{example}

\end{frame}


\begin{frame}
\frametitle{ET0L\phantom{\footcite{rozenberg1973}}}

\begin{example}
	Let $\Sigma = \{a,b\}$ and $V = \{S,A,B\}$, then the following are tables
	\[
	\alpha \, : \,
	\left\{
	\begin{aligned}
	S &\to SS\\
	S &\to S\\
	S &\to AB\\
	A &\to A\\
	B &\to B
	\end{aligned}
	\right.
	\qquad
	\beta \, : \,
	\left\{
	\begin{aligned}
	S &\to S\\
	A &\to aA\\
	B &\to bB
	\end{aligned}
	\right.
	\qquad
	\gamma \, : \,
	\left\{
	\begin{aligned}
	S &\to S \\
	A &\to \varepsilon \\
	B &\to \varepsilon
	\end{aligned}
	\right.
	\]
\end{example}

We then apply tables as 
\begin{multline*}
	S
	\visible<1->{\to^\alpha SS}
	\visible<2->{\to^\alpha SS AB}
	\visible<3->{\to^\alpha AB AB AB}
	\visible<4->{\to^\beta  aA bB aA bB aA bB} \\
	\visible<5->{\to^\beta  aaA bbB aaA bbB aaA bbB}
	\visible<6->{\to^\gamma aa bb aa bb aa bb}
\end{multline*}


\end{frame}


\begin{frame}
\frametitle{Why is ET0L Interesting?}

The solutions to equations\footcite{ciobanu2016a}\textsuperscript{,}\footcite{diekert2016}\textsuperscript{,}\footcite{diekert2017}
\begin{itemize}
\item 
in free monoids (with twisting) is EDT0L

\item
in (virtually) free groups is EDT0L

\item
in hyperbolic groups is ET0L
\textit{(Ciobanu, Elder)}



\end{itemize}

\vspace*{1em}

Primitive element in $F_2$ are EDT0L.\footcite{ciobanu2018}

\vspace*{1em}

There are E(D)T0L languages of intermediate growth.

\end{frame}

%%%%%%%%%%%%%%%%%%%%%%%%%%%%%%%%%%%%%%%%%%%%%%%%%%%%%%%%%%%%%%%%%%%
%%%%%%%%%%%%%%%%%%%%%%%%%%%%%%%%%%%%%%%%%%%%%%%%%%%%%%%%%%%%%%%%%%%
%%%%%%%%%%%%%%%%%%%%%%%%%%%%%%%%%%%%%%%%%%%%%%%%%%%%%%%%%%%%%%%%%%%

%\begin{frame}
%\frametitle{CSPD}
%TODO
%\end{frame}

%%%%%%%%%%%%%%%%%%%%%%%%%%%%%%%%%%%%%%%%%%%%%%%%%%%%%%%%%%%%%%%%%%%
%%%%%%%%%%%%%%%%%%%%%%%%%%%%%%%%%%%%%%%%%%%%%%%%%%%%%%%%%%%%%%%%%%%
%%%%%%%%%%%%%%%%%%%%%%%%%%%%%%%%%%%%%%%%%%%%%%%%%%%%%%%%%%%%%%%%%%%

\begin{frame}
\frametitle{Result}

\begin{theorem}[B, Elder]
	Every finitely generated bounded automata group has an ET0L co-word problem.
\end{theorem}

\end{frame}

%%%%%%%%%%%%%%%%%%%%%%%%%%%%%%%%%%%%%%%%%%%%%%%%%%%%%%%%%%%%%%%%%%%
%%%%%%%%%%%%%%%%%%%%%%%%%%%%%%%%%%%%%%%%%%%%%%%%%%%%%%%%%%%%%%%%%%%
%%%%%%%%%%%%%%%%%%%%%%%%%%%%%%%%%%%%%%%%%%%%%%%%%%%%%%%%%%%%%%%%%%%

\begin{frame}
\frametitle{Some Questions}

\begin{block}{Q: co-ET0L Groups}
	Is there a `nice' classification of co-ET0L groups?
\end{block}

\vspace*{1em}

\begin{block}{Q: co-Context-Free Groups}
	Is there a `nice' classification of co-context-free groups?
\end{block}

\vspace*{1em}

\begin{block}{Q: Word Problems}
	Is there a group with ET0L word problem that isn't virtually free?
\end{block}

\vspace*{1em}

\begin{block}{Q: Language of Geodesics}
Is there a classification of groups with ET0L language of geodesics?
\end{block}

\end{frame}

%%%%%%%%%%%%%%%%%%%%%%%%%%%%%%%%%%%%%%%%%%%%%%%%%%%%%%%%%%%%%%%%%%%
%%%%%%%%%%%%%%%%%%%%%%%%%%%%%%%%%%%%%%%%%%%%%%%%%%%%%%%%%%%%%%%%%%%
%%%%%%%%%%%%%%%%%%%%%%%%%%%%%%%%%%%%%%%%%%%%%%%%%%%%%%%%%%%%%%%%%%%

\begin{frame}[allowframebreaks]{References}
\printbibliography[heading=none]
\end{frame}

%%%%%%%%%%%%%%%%%%%%%%%%%%%%%%%%%%%%%%%%%%%%%%%%%%%%%%%%%%%%%%%%%%%
%%%%%%%%%%%%%%%%%%%%%%%%%%%%%%%%%%%%%%%%%%%%%%%%%%%%%%%%%%%%%%%%%%%
%%%%%%%%%%%%%%%%%%%%%%%%%%%%%%%%%%%%%%%%%%%%%%%%%%%%%%%%%%%%%%%%%%%

\setbeamertemplate{footline}{}
\setbeamertemplate{navigation symbols}{}
\begin{frame}
\begin{center}
	\huge
	\vspace{2em}
	
	\bfseries
	Thank You!
\end{center}
\end{frame}

\end{document}